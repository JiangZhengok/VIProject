%%==================================================
%% conclusion.tex for BIT Master Thesis
%% modified by yang yating
%% version: 0.1
%% last update: Dec 25th, 2016
%%==================================================


\begin{conclusion}
如今,飞速发展人工智能(AI)正在各个领域展露锋芒,AI工作者期望利用AI改善人类生活。移动机器人、无人车和无人机等领域也属于人工智能的一部分。无人载体要实现无人驾驶的目标,通常需要具有自主定位、地图构建、目标识别、智能避障和路径规划等功能。单一传感器不能完成所有的这些功能,所以无人设备上往往搭载多种传感器,以便弥补单一传感器的不足之处。其中,Visual和IMU一般情况下是无人载体上的必备传感器。Visual和IMU在功能上有很多的互补性,将二者融合能够获得高精度的位姿估计。本文正是以视觉和IMU为对象,总结国内外视觉SLAM以及VIO的研究现状,深入研究Visual-IMU的融合算法,同时搭建硬件设备,将硬件设备和软件算法结合起来构建一套多传感融合定位系统。

本文首先详细研究了传感器的数学模型,以及如何选择相机和IMU。然后对相机和IMU的标定原理和方法进行了分析总结,并给出了标定结果。同时研究了硬件同步的方法。之后研究了基于ROS的传感器信息采集方法。然后进入算法部分,首先研究前端算法,包括图像特征的提取与跟踪,特征的异常点剔除方法,IMU预积分理论。然后研究初始化算法,包括对极几何、三角测量、PnP等理论基础,纯视觉初始化流程,以及视觉/IMU联合初始化方法。前端和初始化为后端非线性优化提供初始值,是后端优化的基础。

本文的后续部分对后端优化和闭环进行了深入研究。首先研究了紧耦合非线性滑窗优化方法,定义了优化的状态量以及目标函数,并对各个残差项进行详细的分析。接着为了使滑窗策略不丢失历史信息,研究了边缘化理论,并研究了本系统的边缘化方法。然后本文研究了闭环检测算法,对词袋模型和字典构建进行了细致的研究。最后,针对检测到的闭环帧,研究了如何对闭环帧进行重定位,如何在检测到闭环后对所有的关键帧进行位姿优化,获得全局一致的定位轨迹。

在文章的最后,针对本系统做了两组实验。一组是公开数据集EuRoc实验,另一组是本地校园车载实时实验。实验环境从简单到复杂,从短距离到长距离。两组本地实验的定位精度分别为0.6\%和1.3\%,表明了本系统的可靠性和有效性。

然而本系统仍旧有一些不足之处,故提出以下需要改进的地方:

1.在硬件方面,本系统的硬件同步做的还不完善,目前IMU和相机同步的时间差在毫秒级,而数据集可以达到微秒级。原因在于本系统的硬件同步电路仅仅采用了一片CD4017计数器,没有添加复杂的滤波电路,可能导致触发脉冲的精度不够高和响应时间不够短。

2. 算法方面,本系统算法复杂,代码繁多,计算量大,目前只能在Intel NUC或者个人PC上进行计算,无法移植到小型的嵌入式设备。后续研究需要对算法和代码进行优化,降低整体计算量。

3. 对于磁力计,鉴于磁力计容易受外界环境干扰,测量数据不稳定,外参标定复杂的情况,本系统仅仅使用磁力计进行航向初始化,并未在后端优化中使用。后续将研究如何对磁力计进行动态补偿、滤波,以获得稳定可用的数据,并将其融合在后端优化中。
\end{conclusion}